\documentclass{article}
\title{Paper IEEE-TRO 21-0184: reply to reviewers}
\author{Florent Lamiraux and Joseph Mirabel}
\date{}
\begin{document}
\maketitle

First we would like to warmly thank the reviewers and associate editor for their
time and valuable remarks. Below are answers to each of their comments.
\subsection*{Reviewer 7}
\begin{enumerate}
  \item \textbf{Comment.} The beginning of the main content of the
    paper seems hard to follow for readers not familiar with
    manipulation planning.  I think this could be easily improved by
    adding one or two explanatory images earlier on in the
    paper.\\
    \textbf{Answer.} We agree that the first picture comes very lately in the
    initial version of the paper. To improve the paper, we have moved Figure 2
    at the beginning of the paper to quickly provide an example of manipulation
    planning problem.
\end{enumerate}

\subsection*{Reviewer 9}
\begin{enumerate}
\item \textbf{Comment.} In the modeling of SE(2) and SO(2), why using 2
  coordinates for the angular position (sin and cos)? In 2D, using just one
  coordinate (the unique rotation angle) is not enough?\\
  \textbf{Answer.} Using $(\sin\theta,\cos\theta)$ makes the parameterization
  continuous. We have added a comment in the caption of Table I.

\item \textbf{Comment.} A bit after equation (3), the unnumbered
  equation $q(t)=\cdots$ is the minus sign correct? Should it be the
  operator defined in equation (3)?
  \textbf{Answer.} You are perfectly right. We have fixed the error.

\item \textbf{Comment.} I think that some words are supposed to be
  hyperlinks, like ``here''. They do not work.\\
  \textbf{Answer.} You
  are right, there are many hyperlinks in the paper. We have checked them
  one by one.

\item \textbf{Comment.}  Algorithm 1: Line 6 checks whether any output
  variable of E is already computed by a previous explicit
  constraint. What if two explicit constraints are redundant? For
  instance, if we have two constraints that (in addition to other
  configuration variables) set the same identical constraint to the
  same config variable.  \\
  
  \textbf{Answer.} As stated at the beginning of Section IV.D.1:
  "when an explicit constraint is not successfully added following
  Algorithm 1, it is handled as an implicit constraint." If redundant
  explicit constraints are inserted, the second one will be handled as
  an implicit constraint. By substitution, in the system of implicit
  constraints, some lines will be of the form $0=0$. Those lines will
  be ignored when computing the pseudo-inverse of the Jacobian of the
  implicit system. As a result, the solver will solve both constraints.
  This will only affect the time of computation since the
  size of the Jacobian is bigger than if only the first explicit
  constraint had been inserted. The same is true with redundant
  implicit constraints.

\item \textbf{Comment.} what happens if you detect that a projected
  path is not continuous?\\
  \textbf{Answer.} In this case, only a
  continuous part starting at the beginning of the path is used by the
  M-RRT algorithm. We have added a remark in Section IV.E.1 and in
  Algorithm 4.

\item \textbf{Comment.} Has the final configuration to be fully
  described? E.g., can I provide just a final desired object pose in
  the cartesian space but not specify the corresponding robot inverse
  kinematic solution (that could be not unique)?\\
  \textbf{Answer.}  The M-RRT algorithm as of today is not able to
  define the goal state by some constraints, although it would not be
  difficult to modify it for that. However, the python bindings enable
  users to quickly develop new algorithms where the goal can be
  defined by some constraints. In the construction set example
  described in Section VII.B.2, page 17 and in Figure 13, the goal is
  a state of the constraint graph.

\item \textbf{Comment.} Section V.A Grasp: the equation at the
  beginning. Should it be $C = C_{r_1} \times \cdots \times C_{r_{nr}} \times
  SE(3)^{no}$? (I added the dots).\\
  \textbf{Answer.} You are perfectly right. We fixed the error.

\item \textbf{Comment.} In equation (13) (and also in other
  equations), is $R^3 \times SO(3)$ the same as $SE(3)$?\\
  \textbf{Answer.} They both represent rigid-body transformation but
  they have different Lie group operators. We have added a paragraph
  after Equation (13) to clarify this point.

\item \textbf{Comment.} The constraint that arises from the stable
  contact pose is piece-wise differentiable. While in the previous
  section the constraints are required to be differentiable. Is there
  any issue?\\
  \textbf{Answer.} Piece-wise differentiable is enough. We have replaced all
  instances of "differentiable" by "piece-wise differentiable".

\item \textbf{Comment.} Section D, after a) Number of states, there is an extra :.\\
  \textbf{Answer.} This has been fixed.

\item \textbf{Comment.}   The used PC has 23 MB of RAM and 3GB of cache. Please
  check these numbers.\\
  \textbf{Answer.} This is 32 GB of RAM and 3 GB of cache. This has been fixed.

\item \textbf{Comment.} I think that Figure 9 is not cited in the text.\\
  \textbf{Answer.} Figure 9 is now cited in Section VII.B.1, page 16.
  
\item \textbf{Comment.}  Since the paths are continuous, how is the
  collision checking carried out? Moreover, If I well understand,
  during the waypoint transition the collision checker is not
  called. Is it implicitly assumed here that in such transitions there
  is no collision?\\
  \textbf{Answer.} The software platform proposes discretized
  collision checking as well as continuous collision checking. We have
  added a sentence and a reference in the description of the
  framework (Section VII, page 15). Collisions are checked also along waypoint
  transitions. We have added the word "almost" at the end of Section
  VII-B3, page 18 since the sentence was confusing.
\end{enumerate}

\subsection*{Reviewer 16}
\begin{enumerate}
\item \textbf{Comment.} In the literature survey, it may be good to refer to
  the paper by Lozano-Perez and Kaelbling
  (https://lis.csail.mit.edu/pubs/tlpk-iros14.pdf) because it very much is
  related to representing manipulation planning problems with constraint graphs.\\
  \textbf{Answer.} We have added and cited the reference.

\item \textbf{Comment.} Towards the end of Section III, the paper says
  "We denote by $iq_i$ and $iv_i$ the indices of joint $i$ ...". But there
  are multiple indices of joint $i$ (namely $nq_i$ and $nv_i$ many). So
  maybe the sentence should say "the starting (or first?) indices of
  joint i".\\
  \textbf{Answer.} You are right. This has been fixed (page 4, right column).

\item \textbf{Comment.} At the end of Section IV-B, the paper refers
  to different linesearch methods implemented. It may make sense to
  briefly describe what these different methods do, or give a
  reference next to each, so that the reader can understand what they
  refer to if interested.\\
  \textbf{Answer.} We have added a reference for one of them and explanations
  for the other ones.

\item \textbf{Comment.} In Section IV-E, the paper says the
  constraints are applied only during "path evaluation". It is not
  perfectly clear what is meant by "path evaluation". What does
  "evaluating" a path mean?\\
  \textbf{Answer.} "Path evaluation" means "computing the configuration at a
  given parameter". We have added the definition in the paper.

\item \textbf{Comment.} In Section V, the paper defines the
  manipulation problem, and as admissible configuration properties, I
  wonder if it should also mention being "collision-free" as a
  requirement. It is kind of obvious, but may be good to be
  complete.\\
  \textbf{Answer.} You are right. We have added the absence of collision in the
  definition of admissible configurations.

\item \textbf{Comment.} At the beginning of Section V-A, the combined
  configuration space is defined, but it seems to include only the
  first and the last robot ($C_{r_1}$ and $C_{r_{nr}}$). Where are the other
  robots in between 1 and nr?\\ \textbf{Answer.} You are right. This has
  been fixed.

\item \textbf{Comment.} In Section V-A, the freedom of a grasp seems
  to be defined only using binary true/false flags. Using this method,
  is it possible to define limits? For example, for a cylinder, the
  gripper may be able to slide up and down along the height of the
  cylinder. Is it possible to represent using this approach, or would
  one need to discretize along the height of the cylinder? It would be
  good to state this explicitly (whether possible or not). (This
  capability [e.g. limits on translation on a table] is also what is
  needed for the "Stable contact pose" formalization in Section
  V-B. This raises the question: is there really a need to have two
  separate formalizations for hand-object and object-surface
  attachments, or can there be one unified
  formalization?)\\
  \textbf{Answer.} Allowing a free translation along the axis of a
  cylindrical object is possible using flags. However, the user needs
  to manually add inequality constraints to limit this translation. We
  have added this case in the paper with an explanation (page
  8). Limits of translation are automatically defined for contact
  surfaces by providing plane convex polygons. It seems to us
  difficult to merge these two types of constraints.

\item \textbf{Comment.} One nice trick in Simeon et al. [30] is the
  "virtual closed chain" motions when an object is both on support
  surface and is grasped. Is it possible to have similar solutions
  using the formalization here?\\
  \textbf{Answer.} The constraint graph representation actually
  enables us to generalize the reduction property in [30]. For
  instance, the benchmark "Romeo placard" could be solved this way. To
  do so, we would need to modify the transition constraints when
  building the constraint graph. We have added this task as a future
  work in the conclusion.

\item \textbf{Comment.} In the experimental results section (VII-B) it
  would be good to state the dimensionality of the problems
  (i.e. robot number of DoF and maybe even objects' number of DoF) as
  that would put the planning times a bit more in
  perspective.\\
  \textbf{Answer.} We have added the number of degrees of freedom in the tables
  compiling the experimental results for each benchmark.
\end{enumerate}

\subsection*{Associate editor}
\begin{enumerate}
\item \textbf{Comment.} In addition, given the importance of the
  software framework, the authors are advised to make sure to provide
  a link to the software repositories on github and to give a proper
  visibility in the manuscript.\\ \textbf{Answer.} There was already a
  link to the installation page, but we have made it more visible in
  the introduction (page 1, left column).

\item \textbf{Comment.} Don't use citations as nouns like in "evoked
  in [2]". Rather use "evoked by Mirabel~\textbackslash emph\{et al.\}~\textbackslash cite\{...\}".
  Don't start a sentence with a citation like "[47] propose". This
  would be fixed by the above change.\\
  \textbf{answer.} We have changed citations as requested.

\item \textbf{Comment.}  Overall, this paper is substandard with
  respect to writing. The authors are strongly encouraged to carefully
  revise the writing and spelling of the paper. If this should not be
  corrected in the revised version, this paper will not be recommended
  for publication despite the contributions and this publication
  recommendation. The authors are encouraged to ask a native speaker
  for advise.\\ \textbf{Answer.} %You probably mean "for advice" ?
  We have asked a professional interpreter to read and correct our
  paper. The new version should be perfect.
\end{enumerate}
\end{document}
